\chapter{Závěr}
V této práci jsou diskutovány podmínky rozšíření stávajícího přístupového systému pracujícího s průmyslově standardizovanou sítí RS485 o bezdrátovou senzorovou síť založenou na LoRaWAN protokolu v jednokanálovém módu.
Dále je takováto senzorová síť navržena a realizována vytvořením prototypu jednokanálové LoRaWAN gatewaye připojené do sítě RS485 infrastruktury přístupového systému, kde jsou připojeny tzv. CKP zařízení ovládající dveřní zámky a čtečky, komunikující vlastním proprietárním CKP protokolem, který podoruje i navržená gateway.
K navržené gatewayi jsou připojeny LoRaWAN senzory třetích stran jako koncová zařízení senzorové sítě.
Na univerzitě, kde je tento přístupový systém zaveden, je proveden test dlouhodobého provozu. Při tomto testu je navržená gateway připojena do jedné sítě RS485, která již obsahje 12 CKP zařízení (párů čteček a dveřních zámků).

Byla provedena frekvenční analíza délek paketů a je zvažována největší hodnota délky paketu, stejně jako rezerva rychlosti přenosu dat v síti RS485 za účelem ochrany správné funkcionality systému řízení přístupu.
% Frequency analysis of packet lengths is performed and the biggest value of packet length is considered as well as the reserve of the RS485 data rate in order to protect the access control system from malfunction. 

Z nměřených dat je vypočítán maximální počet koncových zařízení senzorové sítě souběžně odesílajících data, v závislosti na rychlosti přenosu dat v síti RS485 a rezervě přenosové rychlosti, např 81 koncových zařízení připojených ke gatewayi senzorové sítě, napojené do sítě RS485 s přenosovou rychlostí 57600~bps a 10 \% rezervou. 
% This number of sensor nodes significantly exceeds the actual needs of the sensor nodes on one floor block of university building. Therefore we can state that WSN is suitable for smart metering applications.

