% \chapter{The purpose of low power wireless sensor networks and its state of art}
\chapter{Úvod}
Se stále rostoucím využitím IoT (Internet of Things) aplikací v různých oblastech, např. zemědělství, smart metering, smart cities, smart homes, roste i počet dostupných bezdrátových technologií se zaměřením na 
nízkou spotřebu energie, velký dosah, nízkou přenosovou rychlost a nízkou cenu.

IoT aplikace na krátký dosah %, jako je použití v domácnostech (smart home) 
většinou využívají bezdrátové technologie Zigbee nebo Bluetooth, tedy technologie používající pásmo ISM (Industrial Scientific and Medical) 2.4 GHz \cite{Design and Implementation of an IoT Assisted Real Time ZigBee Mesh WSN}, \cite{Internet of Things (IoT) for building Smart Home System}.
IoT aplikace vyžadující dlouhý dosah jsou většinou založené na druhu technologií LPWAN (Low Power Wide Area Network) \cite{A comparative study of LPWAN technologies for large-scale IoTdeployment}. 
Mezi nejpoužívanější technologie LPWAN v pásmu ISM patří LoRa a SigFox a mimo pásmo ISM to jsou NB-IoT (NarrowBand-Internet of Things) a LTE-M (Long Term Evolution-Machine Type Communication).
LPWAN technologie mají dosah 10–15 km v oblastech mimo město a 2–5 km v městských oblastech \cite{Long-Range Communications in Unlicensed Bands}. Topologie LPWAN sítí jsou hvězdice (centralizovaná), hvězdice z hvězdic (decentralizovaná) a mesh (distribuovaná) \cite{high density LPWAN}.
Nízká spotřeba koncových zařízení senzorových sítí jim umožňuje dlouhou životnost z baterie, i delší než 10 let.
Nízké ceny HW (Hardware) je dosaženo úplnou integrací transceiverů s minimálním množstvím komponentů mimo čip \cite{MURS Band for LPWAN Applications}. 


Průmysl IoT roste, protože má obrovský potenciál. Cisco studie \cite{IoT cisco study} uvádí, že IoT bude kombinováno s dalšími technologiemi, jako je umělá inteligence, fog computing a blockchain. Takovéto kombinace technologií vytvoří vyšší hodnotu investice pro firmy.
IoT aplikace v oblasti smart cities požadují škálovatelné pokrytí sítě. Toho je možné dosáhnout propojením několika gatewayí, jak je navrženo v \cite{Flexible Wireless Sensor Network for smart lighting applications}, kde všechny gatewaye jsou připojeny k web serveru přístupným přes internet a cílem tohoto sytému je řídit městské veřejné osvětlení a do budoucna je plánováno použití pro měření (smart metering).
Podobná aplikace je navržena v \cite{Design and Implementation of an IoT Assisted Real Time ZigBee Mesh WSN}, která se zaměřuje na automatické měření ve městě, ale škálovatelný dosah je zajištěn topologií sítě mesh.
IoT aplikace v konceptu budov (smart buildings) jsou běžně navrženy tak, že zařízení v síti jsou připojeny ke cloudu přes Bluetooth gateway nebo Wi-Fi router připojený k internetu \cite{Internet of Things (IoT) for building Smart Home System}.
Podobná aplikace je navržena  v \cite{Building a Smart Home System with WSN and Service Robot}, kde v Zigbee síti jsou zařízení ovládána jedním zařízením typu master, které je připojeno k PC přes RS232.
Základní systém měření (smart metering) je možné vytvořit pomocí zařízení se senzory připojených ke gatewayi senzorové sítě, která je připojena k PC, kde jsou data ze senzorů zaznamenávány a vyhodnocovány \cite{A Meter Reading System Based on WSN}, \cite{Smart Water Meter System for User-Centric Consumption Measurement}, \cite{Radio Data Infrastructure for Remote Monitoring}.
Měřící systémy vyžadující velký dosah mohou být navrženy použitím více gatewayí připojených k serveru, ze kterého jsou pak data získávána aplikačním serverem \cite{Smart Electric Meter Using LoRA Protocols and Iot applications}. 
Na tomto principu funguje např. TTN (The Things Network) \cite{ttn}, což je otevřená LoRaWAN síť, do které uživatelé zdarma připojují vlastní koncová zařízení a gatewaye. Data z koncových zařízení jsou odesílány na TTN server odkud je uživatelé získávají vlastním aplikačním serverem. Uživatel tak může využívat gatewaye TTN sítě instalované jinými uživateli. Pokud potřebuje pokrytí TTN sítě v místě, kde doposud není, může v tomto místě instalovat vlastní TTN gateway a tím rozšířit pokrytí TTN sítě pro všechny uživatele.
% Similar network is proposed in Smart Farm application \cite{Implement Smart Farm with IoT Technology} with the difference that nodes can also be connected to the gateway via RS485 which forms a hybrid wired /wireless system.


Tato práce se zabývá návrhem rozšíření přístupového systému o WSN (Wireless Sensor Network) se zaměřením na nízkou spotřebu energie, která by mohla být využita pro aplikace smart metering a smart buildings v budově a jejím okolí.
Navržená gateway senzorové sítě je připojena do infrastruktury přístupového systému stejně jako zařízení s označením "CKP" (dále jen CKP), které ovládá čtečku a dveřní zámek, tudíž gateway musí podporovat stejný komunikační protokol jako CKP zařízení. 
Toto může vézt ke komplikacím, protože CKP zařízení odesílá pouze krátké pakety, např. paket s ID karty, když je ke čtečce přiložena.
Gateway senzorové sítě je navržena a otestována na přístupovém systému zavedeném v budově univerzity.
Výsledky ukazují, že infrastruktura přístupového systému zvládne až tisíce koncových zařízení senzorové sítě v závislosti na použité přenosové rychlosti v síti RS485 přístupového systému.



%%%%%%%%%%%%%%%%%%%%%%%%%%%%%%%%%%%%%%%%%%%%%%%%%%%%%%%%%%%%%%%%%%%%%%%%%%%%%%%%%%%%%%%
%       NOT USED
%%%%%%%%%%%%%%%%%%%%%%%%%%%%%%%%%%%%%%%%%%%%%%%%%%%%%%%%%%%%%%%%%%%%%%%%%%%%%%%%%%%%%%%

% \cite{Implement Smart Farm with IoT Technology} includes design of hybrid (wired / wireless) smart metering system for multiple farms with multiple LPWAN gateways where nodes are sensors and actuators and communicate via MQTT. As the future work there is said that it can also be used for monitoring grazing livestock by attachin nodes to its bodies and also the system is supposed to be used for studying the development of environmental algorithms for optimization of plants growth with use of environmental data and plant growth data.
% https://ieeexplore.ieee.org/stamp/stamp.jsp?tp=&arnumber=8323908

%%%%%%%%%%%%%%%%%%%%%%%%%%%%%%%%%%%%%%%%%%%%%%%%%%%%%%%%%%%%%%%%%%%%%%%
% not used articles:
% Article \cite{ZigBee-based Vehicle Access Control System} icludes design of vehicle access control system based on Zigbee network which is connected through network coodrinator via RS232 to PC.
% https://ieeexplore.ieee.org/stamp/stamp.jsp?tp=&arnumber=5453569





