\chapter{Návrh vylepšení systému}
\DIFaddbegin \DIFadd{V této kapitole jsou navrženy způsoby vylepšení navrženého systému.
}\DIFaddend 


\section{Možnosti odstranění omezení navržené gatewaye - závislost na znalosti typu koncových zařízení senzorové sítě}
\DIFaddbegin \DIFadd{Jedním z největších omezení navrženého systému je, že FW gatewaye musí podporovat typy všech koncových zařízení senzorové sítě, tudíž přidání nového typu zařízení do sítě vyžaduje update FW gatewaye.
Je zde několik možností jak toto obejít, aby gateway nebyla závislá na znalosti typu koncových zařízení senzorové sítě, ale všechny tyto možnosti zapříčiňují nežádoucí vliv a to zvýšení objemu přenášených dat v síti RS485 přístupového systému.
Původní řešení totiž klade důraz na nízký objem přenášených dat v síti RS485.
}\DIFaddend 

\DIFaddbegin \DIFadd{Například použitý typ koncového zařízení RHF1S001 \ref{sec:RHF1S001} odesílá LoRaWAN pakety o délce 22 bytů, z toho je 9 bytů aplikační payload. Protokol umožňuje posílat pakety o velikosti až 256 bytů, z toho je nejméně 13 bytů data protokolu, tedy maximální délka aplikačního payloadu může být až 243 bytů.
}

\DIFadd{Nejefektivnější řešení by bylo rozšíření CKP protokolu sítě RS485 přístupového sytému o nový typ příkazu umožňující odeslat paket o maximální délce payloadu až 243 bytů, tedy aby jedním tímto paketem bylo možné odeslat celý aplikační payload. 
Je zde bezpečnostní výhoda, že aplikační payload by pak mohl být dešifrován až na serveru LoRaWAN klíčem AppSKey, tudíž by data z koncových zařízení by nebyla odhalitelná v síti RS485.
Toto řešení tedy vyžaduje update FW všech kontrolních panelů, čož je velmi náročné.
%DIF >  tudíž realizace tohot řešení prozatím neni plánována.
}

\DIFadd{Nicméně je možné implementovat podobné řešení bez nutnosti zavádění nového typu paketu pro aplikační payload koncového zařízení senzorové sítě tak, že aplikační payload by byl rozdělen a odeslán posloupností několika paketů průchod. 
Např. z dostupných 6 bytů v jednom paketu průchod by jeden byte obsahoval informaci o čísle paketu dané sekvence a zbylých 5 bytů by byly data paýloadu daného koncového zařízení Zařízení typu RHF1S001, které má aplikační payload o velikosti 9 bytů by byl takto odeslán dvěma pakety průchod. Jelikož každý paket průhod je potvrzován, toto řešení by rapidně zvýšilo objem přenášených dat v síti RS485.
}





\DIFaddend %%%%%%%%%%%%%%%%%%%%%%%%%%%%%%%%%%%%%%%%%%%%%%%%%%%%%%
\section{Návrh gatewaye verze 2}
Pro lepší mechanické uspořádání byla navržena verze 2 \DIFdelbegin \DIFdel{, kde je }\DIFdelend \DIFaddbegin \DIFadd{s navrženým plošným spojem (PCB).
Schéma zapojení je v obrázku \ref{fig:minigateway_schema} a plošný spoj je v obrázku \ref{fig:minigateway_plosnak}.
Je zde }\DIFaddend použit \DIFdelbegin \DIFdel{i }\DIFdelend jiný vývojový kit NUCLEO-L432KC s \DIFaddbegin \DIFadd{výkonnějšim }\DIFaddend procesorem STM32L432KC, který \DIFaddbegin \DIFadd{má pinout stejný jako Arduino Nano, tedy je pod tímto názvem ve shcématu.
LoRa transceiver je použit RFM95w \mbox{%DIFAUXCMD
\cite{RFM95w} }\hspace{0pt}%DIFAUXCMD
bez shieldu.
RS485 transceiver je použit LTC1480. 
Do zařízení }\DIFaddend je \DIFdelbegin \DIFdel{výkonnější.
Navíc je zde }\DIFdelend \DIFaddbegin \DIFadd{dále }\DIFaddend přidán externí stabilizátor, napěťový filtr, \DIFdelbegin \DIFdel{přepínačem }\DIFdelend \DIFaddbegin \DIFadd{přepínač }\DIFaddend volitelné impedanční zakončení sítě RS485 a \DIFdelbegin \DIFdel{proudové ochrany }\DIFdelend \DIFaddbegin \DIFadd{napěťové ochrany pro linky A, B a napájení}\DIFaddend . 

\begin{figure}[!h]
    \centering
    \includegraphics[width=1\textwidth]{minigateway_schema}
    \caption{Návrh \DIFdelbeginFL \DIFdelFL{WSN }\DIFdelendFL gatewaye verze 2 - schéma}
    \label{fig:minigateway_schema}
\end{figure}

\begin{figure}[!h]
    \centering
    \includegraphics[width=0.7\textwidth]{minigateway_plosnak}
    \caption{Návrh \DIFdelbeginFL \DIFdelFL{WSN }\DIFdelendFL gatewaye verze 2 - plošný spoj}
    \label{fig:minigateway_plosnak}
\end{figure}
\DIFaddbegin 

\DIFadd{Použitý procesor neobsahuje paměť EEPROM, tudíž pro ukládání konfigurace gatewaye a zařízení senzorové sítě je použita paměť flash.
}\DIFaddend 



