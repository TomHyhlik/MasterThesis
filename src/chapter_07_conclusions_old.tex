\chapter{Závěr}

% V této práci jsou diskutovány podmínky rozšíření stávajícího systému řízení přístupu pracujícího v průmyslově standardizované síti RS485 s bezdrátovou senzorovou sítí založenou na jednokanálovém režimu LoRaWAN.
%  


% In this paper, the conditions of extending an existing access control system running in an industry standardized RS485 network with a wireless sensor network based on LoRaWAN single-channel mode is discussed. 
% Design of wireless sensor network is performed, i.e., the sensor nodes and one single-channel gateway based on LoRaWAN protocol are designed. The gateway represents a type of CKP device connected to the RS485 network, therefore is supports the existing protocol in the RS485 network. A long-term operation measurement is performed in one university floor infrastructure consisting of twelve CKP devices (pairs of card reader and door lock) and one gateway. Frequency analysis of packet lengths is performed and the biggest value of packet length is considered as well as the reserve of the RS485 data rate in order to protect the access control system from malfunction. 
% Maximum number of wireless sensor nodes simultaneously transmitting data RS485 network is calculated in dependence on RS485 data rate and the reserve of data rate, e.g., 81 sensor nodes that work in RS485 network with a 57600~bps data rate and 10 \% reserve. This number of sensor nodes significantly exceeds the actual needs of the sensor nodes on one floor block of university building. Therefore we can state that WSN is suitable for smart metering applications.

