% \chapter{Measurement Results and Discussion}
\chapter{Měření, výsledky a diskuse}
Tato kapitola zahrnuje testování navržené senzorové sítě v univerzítní budově ČVUT za reálného provozu.

Testování je provedeno v jednom bloku patra univerzity, kde je do jednoho kontrolního panelu připojeno dvanáct CKP zařízení přes síť RS485. Každé z nich ovládá jedny dveře, tedy jednu čtečku a dveřní zámek.
Do této sítě RS485 je navíc připojena navržená gateway jako třinácté CKP zařízení.
K této navržené gatewayi senzorové sítě jsou připojena koncová zařízení se senzory teploty a vlhkosti.
Gateway a CKP zařízení jsou zapojena del blokového schematu v obrázku \ref{fig:ACS architecture IMA with geteway}.
Konkrétní rozmístění stávajcích dvanácti CKP zařízení, gatewaye a dvou koncových zařízení senzorové sítě v testovaných prostorách budovy je zobrazeno v obrázku \ref{fig:CorridorFloorPlan}.

\begin{figure*}[!ht]
    \centering
    \includegraphics[width=0.9\textwidth]{5patro}
    \caption{Rozmístění koncových zařízení sítě a zařízení CKP v testovaných prostorách budovy}
    \label{fig:CorridorFloorPlan}
\end{figure*}

Testování probíhalo od 21. září do 31. října, tedy v době přítomnosti studentů a zaměstnanců v testovaných prostorách. 
Po tuto dobu testování byly zaznamenávány přenášené pakety síti RS485, kontrolní panel přijal 1~876~978 paketů (14~074~522 B) a odeslal 1~101~556 paketů (8~295~219 B), dohromady tedy 2~978~534 paketů (22~369~741 B).
Z naměřených hodnot byla provedena metoda frekvencni analízy. Z přenesených paketů byly nejdelší 3 o velikostí 40 bytů.
S ohledem na celkové množství paketů je to zanedbatelné množství, tj. 1,3E-04 \%.
Avšak vzhledem k povaze systému, tedy ystému s primární funkcí řízení přístupu do omezených oblastí, se za nejhorší scénář považuje nepřekonatelný limit.

Maximální počet koncových zařízení připojených ke gatewayi při kterém není ovlivněn stávájcí přístupový systém je možné vypočítat z rychlosti přenosu dat v síti RS485. 
Tato rezerva rychlosti přenosu dat je uvažována za účelem ochrany přístupového systému před dysfunkcí nebo poruchou, například před dlouhým čekáním na otevření dveří.

\begin{table}[h]
\centering
\footnotesize
\caption{Packet length frequency analysis}
\begin{tabular}{cr}
Packet length &  Count \\ \hline
\textbf{7}  &  \textbf{2~216~098}  \\
8  &   619~127   \\
9  &         3   \\
11 &    58~393   \\
13 &    58~620   \\
16 &         1   \\
18 &         2   \\
\textbf{19} &    \textbf{26~286}   \\
23 &         1   \\
40 &         3   \\
\end{tabular}
\label{tab:FreqAnalysis}
\end{table}

\begin{figure*}[ht]
    \centering
    \includegraphics[width=1\textwidth]{03-dr-measured}
    \caption{Measured data rate in [bps] in RS485 network during long-term operation test}
    \label{fig:PacketLengthMeasuredAll}
\end{figure*}

Na základě frekvenční analízy uvedené v tabulce \ref{tab:FreqAnalysis} a IMA know-how, pakety přenášející data z koncových zařízení jsou dlouhé 19 bytů a pakety potvrzení IMA protokolu jsou dlouhé 7 bytů. Alespoň dva pakety jsou potřeba k přenesení dat z koncových zařízení přes síť RS485, tj. paket s daty koncového zařízení a paket potvrzení.





V obrázku \ref{fig:PacketLengthMeasuredAll} jsou dvě důležité charakteristiky, maximální délka paketu 
(červená přerušovaná čára) a medián délky paketu (červená nepřerušovaná čára), určeny frekvenční analízou v tabulce \ref{tab:FreqAnalysis}. 
Průměrné zatížení provozu kanálu sítě RS485 je 6,38 pps, tj. 0,85 Bps.

%shows lengths of captured packet  during
%Median of packet length is determined by using frequency analysis method, as shown in Tab. \ref{tab:FreqAnalysis}.
% 7,51 Byte --> simple arithmetic mean









During long term operation test, lengths of transmitted packets ($ l $) are captured with the accuracy of timestamps of a thousandth of a second. Then resampled to one second resolution interval using the sum function to easily represent achieved data as a bit rate in bit per second (bps),
Fig.~\ref{fig:PacketLengthMeasuredAll}.
Red colored dashed line (with value of 2520~bps) shows one second time interval in which a sum of captured packets is transported in RS485 network. It shows, based on detailed knowledge of the IMA protocol, less than 20\% of the RS485 network capacity is used. 
%This limit state is caused by data communication of two sensor nodes that occupied 2\% of capacity of used RS485 network.



% During long term operation test, lengths of transmitted packets ($ l $) are captured with the accuracy of timestamps of a thousandth of a second. Then resampled to one second resolution interval using the sum function to easily represent achieved data as a bit rate in bit per second (bps),
% Fig.~\ref{fig:PacketLengthMeasuredAll}.
% Red colored dashed line (with value of 2520~bps) shows one second time interval in which a sum of captured packets is transported in RS485 network. It shows, based on detailed knowledge of the IMA protocol, less than 20\% of the RS485 network capacity is used. 
% %This limit state is caused by data communication of two sensor nodes that occupied 2\% of capacity of used RS485 network.

% To avoid RS485 network congestion the maximum number of sensor nodes $ S_{MAX} $ can be calculated as:
% \begin{equation}
% S_{MAX} = \frac{\frac{\frac{v_{485}}{B}}{l_{MAX}} - R}{P}
% \label{equ:max-count-of-sensors}
% \end{equation}

% where:

% \begin{tabular}{l @{  } l}
% $v_{485}$ & 485 network data rate [bps]\\
%  B        & bits to Byte \\
% $l_{MAX}$ & maximal packet length \\
%  R        & reserve of the data rate [\%]\\
%  P        & number of packets to transmit sensor data \\
% \end{tabular}

% Considering above mentioned limits, desired reserves and RS485 data rates, the maximum number of sensor nodes simultaneously transmitting their data on RS485 network is calculated, Tab. \ref{tab:max-sensor-nodes}.

% Values for calculation are:

% \begin{tabular}{l @{ $=$ } l}
% $v_{485}$ & RS485 network data rate \\
%  B        & 8 \\
% $l_{MAX}$ & 40 \\
%  P        & 2 \\
% \end{tabular}

% % todo: this table makes error
% % \begin{table}[ht]
% % \centering
% % \footnotesize
% % \caption{Maximum number of sensor nodes simultaneously transmitting their data in RS485 network with desired reserve}
% % \begin{tabular}{r|rrrr}
% % \multicolumn{1}{c|}{\textbf{RS485}}     & \multicolumn{4}{c}{\multirow{2}{*}{\textbf{Reserve} $R$}}\\
% % \multicolumn{1}{c|}{\textbf{data rate}} &   \\
% % $v_{485}$ {[bps]}  &	0 \%	&	10 \%	&	20 \%	&	30 \%  \\ \hline
% %   1200~~~ &    1	&    1	&    1	&    1 \\
% %   2400~~~ &    3	&    3	&    3	&    2 \\
% %   4800~~~ &    7	&    6	&    6	&    5 \\
% %   9600~~~ &   15	&   13	&   12	&   10 \\
% %  19200~~~ &   30	&   27	&   24	&   21 \\
% %  38400~~~ &   60	&   54	&   48	&   42 \\
% %  57600~~~ &   90	&   81	&   72	&   63 \\
% % 115200~~~ &  180	&  162	&  144	&  126 \\
% % 230400~~~ &  360	&  324	&  288	&  252 \\
% % 460800~~~ &  720	&  648	&  576	&  504 \\
% % 921600~~~ & 1440	& 1296	& 1152	& 1008 \\
% % \end{tabular}
% % \label{tab:max-sensor-nodes}
% % \end{table}

% For example, WSN can connect up to 162 sensor nodes that work in RS485 network with a 115200~bps data rate and 10 \% reserve, or up to 126 sensor nodes with a 115200~bps data rate and 30 \% reserve. The results also show, one floor block of university building, i.e., one RS485 network, can operates dozens of sensors with sufficient reserve protecting the access control system from malfunction.   

















%%%%%%%%%%%%%%%%%%%%%%%%%%%%%%%%%%%%%%%%%%%%%%%%%%%%%%%%%%%%%%%%%%%%%%%%%%%%%%%%%%%%%%%
%       NOT USED
%%%%%%%%%%%%%%%%%%%%%%%%%%%%%%%%%%%%%%%%%%%%%%%%%%%%%%%%%%%%%%%%%%%%%%%%%%%%%%%%%%%%%%%
% The heavy traffic test simulates data transmission in the RS485 network as evidence of theoretically calculated values as shown in Tab \ref{tab:max-sensor-nodes}.

% \begin{figure}[!ht]
    % \centering
    % \includegraphics[width=.5\textwidth]{03-tp-simul}
    % \caption{RS485 datarates in simulation of heavy data traffic}
    % \label{fig:heavySimulation}
% \end{figure}

% This test simulated the transmission of more than 190~000 commands from 300 sensor nodes every 5 minutes simultaneously for 12 hours time period. The highest datarate achieved during simulation is 1160~bps, Fig \ref{fig:heavySimulation} red dashed line.

%!!! Tady to mozna chce frekvencni analyzu GW, at vis, jak jsou dlouhe pakety, pak nemusis hadat, nebo napis, ze max. delka je 19 bytů.

%Considering the worst case, ie packets with length of 40 Bytes, we have analytically calculated the maximum number of sensors a network can transmit based on the network transmission rate.

%\textbf{!!! Limit RS485: up to 32 transceivers on the serial bus !!! My vsak mame senzory pres LoRa ...Jo, ale to je fyzicky, to splnujeme, mame jich 13 :-)}
%datasheet: https://www.sparkfun.com/datasheets/Components/General/sp3485CN-LTR.pdf

%Space for peaks 10\% --> maximum packet rate is 324 pps
%Data measured in testing procedure shows Fig.

%\begin{table}[h]
%\centering
%\footnotesize
%\caption{Simple analytics of measured data}
%\begin{tabular}{lr}
%\textbf{Packet length} & \textbf{Bytes} \\ \hline
%Minimum   &   7 B     \\
%Maximum   &  40 B     \\
%mean      &   7,51 B  \\
%Median    &   7 B     \\
%\end{tabular}
%\label{tab: simple-analytics}
%\end{table}


%---
% tabulka, rezervy 10%, 30% ...
% Zjistit kolik sensoru muze byt na sbernici 485.
% popis os co  je co
