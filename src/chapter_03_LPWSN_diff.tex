  \chapter{Výběr bezdrátové technologue pro senzorovou síť}
Navržená senzorová síť je napojena na infrastrukturu zavedeného přístupového systému v budově zákazníka s dosahem po celé budově a jejím okolí. Takto navržená síť umožní v těchto prostorách snímání dat z desítek senzorů.
Mezi hlavní kritéria pro vybranou bezdrátovou technologii patří nízká spotřeba energie koncových zařízení, nízká cena, jednoduchost implementace a možnost připojení koncových zařízení třetích stran.
Pro jednoduchost implementace vybraná bezdrátová technologie tedy musí používat pouze bezlicenční pásmo ISM a musí umožňovat uplementaci celé sítě bez závislosti na siti třetích stran. 

\section{Kandidátní bezdrátové technlogie}
Níže jsou popsány dostupné bezdrátové technologie vyhovující stanoveným kritériím pro tento projekt.

\subsection{IQRF}
IQRF je technologie vyvinuta IQRF aliancí \cite{iqrf_alliance}, která je jediným výrobcem IQRF transceiveru \cite{iqrf_transceivers} za cenu v rozsahu \$15-20 za kus a k tomu poskytuje nástroje jako je SDK \cite{iqrf_sdk} a IDE \cite{iqrf_ide}.
Technologie IQRF bývá použita k realizaci sítí o topologii typu mesh nebo hvězdice.
Jedna síť má jednoho koordinátora, který slouží jako gateway a může obsahovat až 240 zařízení (včetně koordinátora). Pro požadavek vyššího počtu zařízení je efektivnější zřetězit více sítí, s jednotlivými koordinátory a různými RF kanály pro vyšší propustnost.
% https://iqrf.org/support/networkingfaq
Dosah na přímou viditelnost je až 500 m a velikost payloadu jednoho packetu může být až 64 B.
Z hlediska kriérií pro tento projekt je u této technologie nevýhodou nízký počet zařízení třetích stran dostupných na trhu. 
Většinou je tato technologie použita pro realizaci uživatelsé sítě, kde gateway je použita jedna z dostupných od IQRF aliance a koncová zařízení sítě jsou vytvořena vývojáři s použitím transceiverů od IQRF aliance
\cite{paper_iqrf}.


\subsection{Wireless M-Bus}
Wireless M-Bus (Meter-Bus) je standard specifikovaný v evropské normě EN 13757, popisující fyzickou, síťovou a aplikační vrstvu, původně navržen pro aplikace bezdrátového měření jako rozšíření průmyslové datové sběrnice M-Bus \cite{wirelessMBus_automatizace}.
Po několika letech v průmyslu bezdrátových měřících systémů se tato technologie rozšířila do oblasti průmyslu senzorových sítí.
Komunikace koncových zařízení je rozdělena do několika módů v závislosti na orientaci komunikace a objemu vysílaných dat \cite{wirelessMBus01} \cite{wirelessMBus02}. 
Wireless M-Bus síť má hvězdicovou topologii o dosahu 500 m v pásmu 868 MHz.
Komunikaci vždy zahajuje koncové zařízení, které koncentrátor obsluhuje.
Transceivery vyrábí více různých firem za cenu okolo \$25-30 za kus.


\subsection{LoRa}
LoRa (Long Range) je modulace navržena firmou Semtech, LoRa Alliance vytvořila síťový protokol s názvem LoRaWAN postaven na fyzické vrstvě LoRa.
Síť má hvězdicovou topologii, kde komunikaci zahajuje koncové zařízení a koncentrátor ho obsluhuje.
Pro nízkou spotřebu protokol umožňuje ladit SF (Spreading Factor) neboli přenosovou rychlost, pro regulaci dosahu, který je až 15-22 km mimo městskou část a 3-8 km ve městské části \cite{lorawan_specification}.
V některých oblastech některé firmy poskytují pokrytí LoRaWAN sítí a proto je tato technologie velmi populární, Na trhu je mnoho koncových zařízení i gatewayí, s jejichž vzájemnou kompatibilitou neni problém. Gateway přeposílá přijaté packety s daty z koncových zařízení na server, kde je payload zpracován na základě dokumentace od výrobce daného koncového zařízení.
Semtech je jediným výrobcem integrovaných obvodů podporujících LoRa modulaci. Na trhu je dostupných mnoho transceiverů, které používají tento integrovaný obvod, některé dokonce obsahují implementovaný LoRaWAN stack.
Transceiver pro gateway má cenu okolo \$130 a umožňuje současně přijímat packety od koncových zařízení na více kanálech a přenosových rychlostech. Je zde i možnost udělat jednokanálovou gateway, která je schopna přijímat v jednu chvíli pouze jednom kanále a jedné přenosové rychlosti s použitím transceiveru pro koncová zařízení za cenu okolo \$5-20. V takové sítí pak musí být všechna koncová zařízení nakonfigurována na jednu konkrétní frekvenci a přenosovou rychlost.


% \begin{figure}[!h]
%     \centering
%     \includegraphics[width=1\textwidth]{spreading_factor_lorawan_2017-07-29}
%     \caption{LoRa spread factor options \cite{24}}
%     \label{fig:loraSF}
% \end{figure}

% \cite{19} \cite{20} \cite{21} \cite{22} \cite{23} \cite{24}.


\subsection{Zigbee}
Zigbee je specifikace navržena pro IoT aplikace, založena na standardu  IEEE 802.15.4, vyvinuta Zigbee alliancí \cite{Zigbee_alliance}.
Technologii Zigbee podporuje topologie mesh a hvězdice, většinou je použita topologie mesh pro rozšíření dosahu sítě, který je mezi dvěma zařízeními do 300 m na přímou viditelnost a 75 až 100 m v budově \cite{Zigbee_alliance_solution}. Jedna síť může obsahovat až 65000 zařízení.
Dostupné transceivery na trhu se pohybují okolo \$8–30 od více různých výrobců, taktéž je i na trhu dostupnýchmnoho Zigbee koncových zařízení.


\subsection{BLE}
Bluetooth Low Energy (BLE) je verze Bluetooth navržena pro minimální spotřebu energie, podporující topologie point-to-point, broadcast a mesh \cite{BT_alliance}.
Dosah mezi dvěmi zařízeními je až 100 m \cite{BT_nordic}.
Dostupné transceivery na trhu se pohybují okolo \$5–20 od více různých výrobců. Stejně tak je i na trhu mnoho dostupných koncových zařízení, ale mnohdy jsou tato koncová zařízení kompatibilní pouze se zařízení v rámci jednoho výrobce, tudíž může být problém je implementovat do vlastní senzorové sítě. 



% \subsection{Z-Wawe}
% Z-Wave is intended for wireless connectivity for all possible smart home products, controlled by PC, phone, voice, etc. It's based on mesh network topology so every non-battery powered device works as a router to enhance the network range so the more devices are connected in one network, the stronger the network is \cite{27} \cite{28}.
% \subsection{Thread}
% This technology based on IPv6 was developed for home network controlled by smartphone, tablet or PC \cite{29} \cite{30} \cite{31}.


\section{Shrnutí vlastností vybraných technologií}
V tabulce \ref{table:shrnutiTechnologii} jsou shrnuty vlastnosti zmíněných technologií.   


% \begin{tabular}{|p{1.5cm}|p{2cm}|p{2cm}|p{2cm}|p{2cm}|p{2cm}|}

\begin{table}[!h]
  \centering
  \begin{tabular}{|p{1.5cm}||p{1.5cm}|p{1.5cm}|p{1.5cm}|p{1.5cm}|p{2.5cm}|}
      \hline
                           & \textbf{IQRF}         & \textbf{Wireless M-bus} & \textbf{LoRa}                 & \textbf{ZigBee}                          & \textbf{BLE}                                              \\ \hline \hline
    topology               & mesh, star            & star                  & star                            & mesh, star                               & Point-to-Point, Broadcast, Mesh                           \\ \hline
    maximum size of packet & 64 B                  & 256 B                 & 256 B                           & 133 B                                    & 20 B (Bluetooth 4.0), 251 B (Bluetooth 4.2)               \\ \hline
    data rate              & 19.2 kb/s             & 32.768 – 100 kb/s     & 0.3-50kb/s                      & 250 kb/s                                 & up to 1 Mb/s                                              \\ \hline
    band (Europe)          & 868 MHz               & 868 MHz               & 868 MHz                         & 2.4 GHz                                  & 2.4 GHz                                                   \\ \hline
    range                  & 500 m (line of sight) & 500 m (line of sight) & 15-22 km suburban, 3-8 km urban & 300 m (line of sight), 75-100 m (indoor) & 100 m (class 1), 10 m (class 2), less then 10 m (class 3) \\ \hline
    security               & AES-128               & AES-128               & AES-128                         & AES-128                                  & AES-128                                                   \\ \hline
    max radio output power & up to 8 mW            & 0.16–20 mW            & 24 mW                           & 1-100 mW                                 & 100 mW (class 1), 5 mW (class 2), 1 mW (class 3)          \\ \hline
  \end{tabular}
  \caption{Souhrn porovnání parametrů kandidátní bezdrátových technologií}
  \label{table:shrnutiTechnologii}
\end{table}


\section{Vybraná přenosová technologie}
% \section{Wireless Sensor Network Design}
Wireless sensor network design is based on a popular IoT technology LoRa, which is a LPWAN technology using ISM band, 433 MHz, 868 MHz and 915 MHz (depends on the region) and communicates on multiple frequency channels and uses multiple data rates \cite{LoRaWAN Evaluation for IoT Communications}.
The LoRaWAN is an open standard network protocol and system architecture specified by \cite{LoRaWAN specification} and creates a media access control (MAC) layer on the top of the LoRa physical layer, secured by AES-128 encryption.
The LoRaWAN nodes communicate directly with the LoRaWAN gateway \cite{Internet of Things (IoT) using LoRa technology}.


