% \chapter{The purpose of low power wireless sensor networks and its state of art}
\chapter{Introduction}

Wireless sensor networks (WSN) 



\section{About LPWAN}
\textit{"Low Power Wide Area Networks (LPWAN) are the
evolution of wireless sensor networks (WSN) for long
range Internet of Things (IoT) applications."} \cite{MURS Band for LPWAN Applications}
This kind of networks differ from custom wireless sensor networks (WSN) with focus on low power, low cost, scalability and extended range.
It has no such emphasis on high data rates and low latency.
The very low power performance should allow sensor nodes very long battery life, even greater than 10 years.
The low cost of HW is being reached by fully integrated transceivers and minimizing number of off-chip components \cite{MURS Band for LPWAN Applications}.


Many new LPWAN technologies appeared at the market in recent years.
Most of them use ISM band, which is 2.4 GHz for short range and 915/868 MHz (depends on region) and 433 MHz to achieve longer communication range.
The main 2.4 GHz technologies are Zigbee and BLE.
The main 915/868/433 MHz technologies are LoRa, SigFox and IQRF.
There are also LPWAN technologies using licensed bands based on LTE standard
such as NB-IoT and CAT-M.

todo: bluetooth a zigbee je LPWAN or LPPAN? patri sem vubec?




LPWAN can have one of the following topologies: star (centralized), star of stars (decentralized) and mesh (distributed).
\cite{high density LPWAN}










\section{The expected LPWAN growth in the future}
The industry of LPWAN in concept of IoT is growing due to its huge potential.
Cisco study \cite{IoT cisco study} says that IoT will be combined with other technologies such as artifical inteligence (AI), fog computing and blockchain. Such combination of technologies will provide greater value of investment for companies. 
The IoT security becomes one of the most relevant requirements.
More organizations will become to cooperate with each other in solution development.
More open standards, open architectures and regulations is to come in the future.
todo: popsat vic jak se WSN vyviji, pripadne najit dalsi studii.





todo: Pak psat o implementacich WSN, tedy neco jako:
"V clanku [1] je WSN pripojena pres RS485 sit k app serveru, vysledkem tohoto reseni je..., v clanku [2] je WSN implementovana takhle..."
% bluetooth WSN to RS485
% https://ieeexplore.ieee.org/stamp/stamp.jsp?tp=&arnumber=8323908
% LoRa to Ethernet/3g
% https://ieeexplore.ieee.org/stamp/stamp.jsp?tp=&arnumber=8125884
% Zigbee to RS232
% https://ieeexplore.ieee.org/stamp/stamp.jsp?tp=&arnumber=5453569











\chapter{Identifikace problému, který WSN řeší}
todo: The WSN brings many advantages...























