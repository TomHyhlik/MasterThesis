\chapter{Účel LPWAN v kontextu současného stavu ve světě}

\section{About LPWAN}
\textit{"Low Power Wide Area Networks (LPWAN) are the
evolution of wireless sensor networks (WSN) for long
range Internet of Things (IoT) applications."} \cite{MURS Band for LPWAN Applications}
This kind of networks differ from custom wireless sensor networks (WSN) with focus on low power, low cost, scalability and extended range.
It has no such emphasis on high data rates and low latency \cite{MURS Band for LPWAN Applications}.


The requirements for LPWAN technologies are to enable end network devices to 
todo: vybrat info z: \cite{MURS Band for LPWAN Applications}.





LPWAN can have one of the following topologies: star (centralized), star of stars (decentralized) and mesh (distributed).
\cite{high density LPWAN}

Many new LPWAN technologies appeared at the market in recent years.
Most of them use ISM band, which is 2.4 GHz for short range and 915/868/433 MHz (depends on region) to achieve longer communication range.
The main 2.4 GHz technologies are Zigbee, Bluetooth and Z-Wawe.
The main 915/868/433 MHz technologies are LoRa, SigFox and IQRF.
There are also LPWAN technologies using licensed bands based on LTE standard
such as NB-IoT and CAT-M.
















\section{The expected LPWAN growth in the future}
The industry of LPWAN in concept of IoT is growing due to its huge potential.
Cisco study \cite{IoT cisco study} says that IoT will be combined with other technologies such as artifical inteligence (AI), fog computing and blockchain. Such combination of technologies will provide greater value of investment for companies. 
The IoT security becomes one of the most relevant requirements.
More organizations will become to cooperate with each other in solution development.
More open standards, open architectures and regulations is to come in the future.
todo: popsat vic jak se IoT vyviji, pripadne najit dalsi studii.
















\chapter{Identifikace problému, který LPWAN řeší}