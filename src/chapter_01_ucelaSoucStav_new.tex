% \chapter{The purpose of low power wireless sensor networks and its state of art}
\chapter{Introduction}
There is a big growth in demand and use cases of IoT applications including security, asset tracking, agriculture, smart metering,smart cities, and smart homes
as well as the growth of IoT wireless technologies, which require long range, low power consumption, low data rate and low cost.
For short-range IoT applications like smart homes is widely used Zigbee or Bluetooth which use ISM band 2.4 GHz.
% todo: zde mozna dopsat neco vic o short-range technologies
For long range applications is formed a new type of technologies called low power wide area network (LPWAN) \cite{A comparative study of LPWAN technologies for large-scale IoTdeployment}, with range up to 10–15 km in rural areas and 2–5 km in urban areas \cite{Long-Range Communications in Unlicensed Bands} and can have one of the following topologies: star (centralized), star of stars (decentralized) and mesh (distributed) \cite{high density LPWAN}.
The very low power performance should allow sensor nodes very long battery life, even greater than 10 years.
The low cost of HW is being reached by fully integrated transceivers and minimizing number of off-chip components \cite{MURS Band for LPWAN Applications}.
Many new LPWAN wireless communication technologies appeared in recent years with licensed and unlicensed band.
The most widely used unlicenced technologies are LoRa and SigFox, using ISM band, 915 or 868 MHz (depends on region) or 433 MHz. These ISM bands are limited to allow these technologies to transmit only up to hundreds of packets per device per day.
 % \cite{A comparative study of LPWAN technologies for large-scale IoTdeployment}.  
The most widely used licensed technolologies are based on LTE and these are NB-IoT and CAT-M.

The industry of IoT is growing due to its huge potential.
Cisco study \cite{IoT cisco study} says that IoT will be combined with other technologies such as artifical inteligence (AI), fog computing and blockchain. Such combination of technologies will provide greater value of investment for companies. 
The IoT security becomes one of the most relevant requirements.
More organizations will become to cooperate with each other in solution development.
More open standards, open architectures and regulations is to come in the future.
% todo: popsat vic jak se WSN vyviji, pripadne najit dalsi studii.

The WSN based IoT applications are being implemented in various architecture combinations.
Article \cite{Implement Smart Farm with IoT Technology} includes design of hybrid (wired / wireless) smart metering system for multiple farms with multiple LPWAN gateways where nodes are sensors and actuators and communicate via MQTT. 
As the future work there is said that it can also be used for monitoring grazing livestock by attachin nodes to its bodies and also the system is supposed to be used for studying the development of environmental algorithms for optimization of plants growth with use of environmental data and plant growth data.
Another article \cite{Radio Data Infrastructure for Remote Monitoring} describes design of location measurement IoT application based on LoRa, where a LPWAN is established by LoRa receiver plugged into PC via USB port, so the sensor data are directly received and analyzed in PC.
Article \cite{ZigBee-based Vehicle Access Control System} icludes design of vehicle access control system based on Zigbee network which is connected through network coodrinator via RS232 to PC.

None of these mentioned articles describe implementation of WSN into any existing system. 
This article aims to extend access control system with IoT which could be useful for implementation of smart building applications such as controlling of lights, heating and so on and measuring temperature, humidity, CO2 and so on.



% \chapter{newwwww shshshs}


% \chapter{Identifikace problému, který WSN řeší}
% % todo: 
% The WSN brings many advantages...























